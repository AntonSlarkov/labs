\documentclass[a4paper, 12pt]{article} 
\usepackage[utf8]{inputenc}
\usepackage{ragged2e}
\usepackage{amsmath, amssymb,graphicx,wrapfig}
\usepackage{listings}
\usepackage[T2A]{fontenc} 
\usepackage[english, russian]{babel}
\usepackage[top = 20 mm, 
            bottom = 20 mm, 
            left = 30 mm, 
            right = 30 mm]{geometry}
\usepackage{amsmath}
\usepackage{graphicx}
\usepackage[colorlinks=true, allcolors=blue]{hyperref}
\begin{document}
УДК 004.912
\begin{center}
    ИНФОРМАЦИОННЫЕ ОСНОВЫ ВЫДЕЛЕНИЯ АББРЕВИАТУР И\\ ИХ РАСШИФРОВКИ В ТЕКСТЕ НА РУССКОМ ЯЗЫКЕ\\
Ивнов И.И., Петров П.П., Васильев В.В.\\
ФГКВОУВО «Академия ФСО России»,\\
Россия, Орёл
\end{center}
\indentАннотация. Данная работа описывает теоретические основы автоматизированного выделения аббревиатур и определения их расшифровок в текстах на русском языке. Изложено текущее состояние предметной области, рассмотрены существующие программные средства автоматизированного составления списка аббревиатур и их расшифровок. Приводится классификация аббревиатур по структурно-информационным признакам. Предложенная классификация подходит для создания алгоритмов автоматизированного выделения аббревиатур. Разработан подход к определению расшифровки аббревиатур без их непосредственного введения в текст.\\
\indent Ключевые слова: аббревиатура, расшифровка аббревиатуры, выделение аббревиатур, классификация аббревиатур.\\
\indent\textbf{1. Вводные положения}\\
\indentРазвитие аббревиации и использование сокращенных лексических единиц – общая тенденция для многих алфавитных языков. Так, аббревиатуры широко используются не только в специализированных областях знания, но и в повседневной коммуникации [1].\\ 
\indentВведём ряд определений. Под \textit{аббревиатурой} будем понимать \href{https://ru.wikipedia.org/wiki/%D0%A1%D0%BB%D0%BE%D0%B2%D0%BE}{слово}, образованное сокращением слова или словосочетания, читаемое по названию начальных букв или по начальным и крайним (общепринятые аббревиатуры) звукам слов, входящих в него. Под \textit{расшифровкой} или \textit{полной формой аббревиатуры} будем понимать последовательность слов, от которых образована аббревиатура. \textit{Введением аббревиатуры в текст} является определенная последовательность аббревиатуры и её расшифровки в одном предложении текста. \textit{Аббревиатурой без расшифровки} является аббревиатура, не имеющая расшифровки в предложении, где она расположена. Под \textit{выделением аббревиатуры (расшифровки)} будем понимать получение структурной информации об аббревиатуре (расшифровке) для её дальнейшего использования.\\
\indentУпотребление аббревиатур – специфическая особенность научно-технических текстов, в которых аббревиатурам принадлежит большая доля информационной нагрузки [2]. В текстах художественного стиля практически отсутствуют аббревиатуры в виду отсутствия ёмких терминов, которые необходимо сокращать.
В научно-технических текстах на русском языке из различных областей знаний используются разнообразные аббревиатуры, что затрудняет возможность интуитивной расшифровки аббревиатуры человеком, читающим текст. При отсутствии \textit{списка аббревиатур и соответствующих им расшифровок} (САиР) для текста возникают трудности с интерпретацией аббревиатур. При этом в большинстве текстов достаточно информации для того, чтобы по определенным признакам восстановить или создать данный список. Это утверждение положено в основу данной статьи. Для отдельных текстов информации, содержащейся внутри них, может быть недостаточно, чтобы восстановить САиР. Тогда следует прибегать к анализу других текстов схожей тематики.\\
\indentПотребность в восстановлении САиР может возникнуть при решении широкого класса задач обработки текстов. Здесь можно упомянуть межъязыковые преобразования текстов, в том числе – конверсию графических систем письма [3], расчет характеристик сложности текста [4], автоматизированное извлечение ключевых слов [5], рерайтинг [6] и квалиметрический анализ текста [7]. Во всех перечисленных задачах необходимо произвести предварительное выделение аббревиатур и их расшифровок (ВАиР) из текста.
Исходя из результатов исследования тексты на русском и английском языках можно разделить на три типа (рисунок 1):\\
\indent1. Тексты, в которых присутствует САиР, \textit{введения аббревиатур и аббревиатуры без расшифровок}.\\
\indent2. Тексты, в которых присутствуют \textit{введения аббревиатур и аббревиатуры без расшифровок} (введённые ранее).\\
\indent3. Тексты, которых присутствуют только \textit{аббревиатуры без расшифровок} (ранее не были введены).\\
\indentПредполагается, что результаты исследования будут актуальны и для других алфавитных языков. Также установлено, что тексты научно-технического стиля более насыщены аббревиатурами, чем тексты аналогичного объёма художественного стиля.\\
\begin{figure}[!h]
    \centering
    \includegraphics[scale=0.5]{image.jpg}
    \caption{Типы текстов с аббревиатурами (разработано авторами)}
    \label{claster_pic1}
\end{figure}
\\
\indent\textbf{2. Состояние предметной области}\\
\indentВ работах [8-10] представляются различные классификации аббревиатур, но не предложено решений по выделению их из текстов.\\
\indentВ работе [11] предложено решение для нахождения полного названия журнала по его аббревиатуре. Неизвестная аббревиатура, указанная пользователем, приводилась в формат регулярного выражения, которое предполагало возможный набор слов, начинающийся с указанных букв. Специфическая реализация полученного решения не позволяет использовать его для определения полных форм аббревиатур из других предметных областей.\\
\indentВ работах [12, 13] на основании определения частот встречаемости соседних слов определяется мера их связности, что позволяет предложить вероятные полные формы аббревиатур. Достоинством такого метода является его универсальность, недостатком – высокая трудоёмкость. Материалы данных работ использовались при разработке модели процесса выделения аббревиатур.\\
\indentВ работах [14, 15] предложено исходный текст представлять в виде совокупности тем, которые образуются множеством входящих в них с разной вероятностью слов. Найденная схожесть частей текста используется как представление полной формы аббревиатуры. Данных подход предлагает множество решений с близкими вероятностями, что предусматривает дополнительную работу для пользователя на стадии отбора расшифровки интересующей аббревиатуры.\\
\indentСуществуют программы для ЭВМ, зарегистрированные в Федеральной службе по интеллектуальной собственности (Роспатент), обладающие возможностью выявления САиР. Так, программа [16] реализует функцию автоматизированного формирования перечня аббревиатур, решает задачу формирования единой базы терминов (аббревиатур) и их определений (расшифровок). Программа [17] предназначена для автоматизированного извлечения терминологических структур из монографии заданной предметной области. Одной из основных функций программы является извлечение терминов, в частности, расшифровка аббревиатур.\\
\indent\textbf{3. Классификация аббревиатур}\\
\indentФормирование классификации аббревиатур осложнено особенностями их структуры, большой вариативностью, множеством различных способов аббревиации, а также взаимодействием аббревиации с другими способами словообразования. Исследователи [10, 18, 19] сходятся во мнении, что аббревиатуры можно подразделять на инициальные, сложносокращённые и общепринятые. В первом случае аббревиатура составляется из первых букв её расшифровки. Во втором случае в аббревиатуру включены не только первые, но и другие буквы сокращаемых слов [20]. В третьем случае аббревиатуры имеют уникальное представление в тексте и единственную расшифровку. Общепринятые аббревиатуры, как правило, интуитивно понятны и употребляются перед определёнными структурами в тексте.\\
\indentДля решения задачи автоматизированного ВАиР из текста введём классификацию по структурно-информационным признакам, а также приведем в первом приближении их распространенность, изученную на материале ста случайно отобранных статей с ресурса Cyberleninka.ru. Аббревиатуры разделяются на три класса: инициальные, общепринятые и сложносокращённые. Инициальные и общепринятые аббревиатуры имеют выраженную структуру (прописные буквы, знаки препинания), которой сложносокращённые не обладают (структурный признак). При этом, общепринятые аббревиатуры имеют интуитивно понятный смысл, а инициальные требуют расшифровки в тексте (информационный признак). Сложносокращённые аббревиатуры, не имеющие в составе прописных букв (завхоз, ликбез и т.д.), рассматриваться в данной статье не будут. Инициальные аббревиатуры разделены на пять типов, каждый из которых отличается по структурным признакам (рисунок 2).
\begin{figure}[!h]
    \begin{center}
        \includegraphics[scale=0.5]{image1.jpg}
        \caption{Классификация аббревиатур по структурно-\\ \centering{информационным признакам (разработано авторами)}}
        \label{claster_pic1}
    \end{center}
\end{figure}
\\
\indentДля создания программного средства ВАиР необходимо учитывать особенности каждого класса рассматриваемых аббревиатур. 
Особенности инициальных и общепринятых аббревиатур:\\
\indent1. (тип А, 53\%) Инициальная аббревиатура, в которой слова полной формы разделены только пробелами и в неё входят только первые буквы слов полной формы. Например: центр информационной безопасности (ЦИБ), Latent Dirichlet Allocation (LDA).\\
\indent2. (тип B, 5\%) Инициальная аббревиатура, в которой некоторые слова полной формы объединены знаком дефис или символом «косой черты». Например: оптико-тепловизионный комплекс (ОТК), read-only memory (ROM), input/output (IO).\\
\indent3. (тип C, 22\%) Инициальная аббревиатура с элементами сложносокращённых слов. При этом, аббревиатура может состоять не только из прописных букв, но первая буква полной формы должна соответствовать первой букве аббревиатуры. Количество слов в расшифровке не совпадает с количеством букв в аббревиатуре. Например: гидрометеорологическая станция (ГМС), ammonium bifluoride (ABF), Белорусский автомобильный завод (БелАЗ), временно исполняющий обязанности (ВрИО).\\
\indent4. (тип D, 5\%) Инициальная аббревиатура, отличная от языка документа. Например, протокол передачи файлов (FTP), временный идентификационный номер подвижного абонента (TMSI).\\
\indent5. (тип E, 2\%) Инициальная аббревиатура, в которой буквы аббревиатуры разделены точками, а первые буквы слов полной формы соответствуют буквам в аббревиатуре. Например: Фамилия Имя Отчество (Ф.И.О.), Петроградская сторона (П.С.).\\
\indent6. Общепринятые аббревиатуры (13\%), которые применяются в разных областях: адреса (г., ул., д., пр-т), звания (к-т, л-т), точные науки (см, Гц), время суток (a.m, p.m), элемент текста (P.S.) и т.п. Они не имеют полной формы в тексте и будут расшифровываться по словарю.\\
\indentДля примеров были использованы аббревиатуры на русском и английском языках, но предполагается, что данная классификация актуальна и для других алфавитных языков.\\

\indent\textbf{4. Модель процесса выделения аббревиатур и расшифровок из текста}\\
\indentПроцесс выделения аббревиатур и их расшифровок в общем может состоять из двух этапов.\\
\indent\textbf{Первый этап} заключается в разделении исходного текста на предложения. Он необходим для более точного определения расшифровок аббревиатур. Разделителем предложений в тексте могут являться восклицательные и вопросительные знаки, многоточия, знаки переноса строки и точки. Однако возникает ряд проблем, связанных с тем, что точки ставятся в тексте не только в конце предложения. Чаще всего точки можно встретить в следующих конструкциях: в датах (25.10.20 г.), в адресах (ул. Ленина, д. 7), в общих аббревиатурах (т.д.), в буквенно-цифровых обозначениях (66.КП.ВРБ.00.00.00.ВО), перед номерами телефонов (тел. 89997773737), в нумерации (1.1, 1.2, …), в составе сокращения ФИО (А.А. Иванов) и в инициальных аббревиатурах (R.I.S.K.).\\
\indent\textbf{Второй этап} разбивается на две параллельных части: \textit{поиск мест введения аббревиатур} и \textit{поиск аббревиатур без расшифровок} в предложениях.\\
\indentПоиск \textit{мест введения аббревиатур} заключается в анализе предложения на предмет наличия аббревиатуры и соответствующей расшифровки. В случае успеха информация о расшифровке и соответствующей аббревиатуре заносится в базу данных. Одна аббревиатура вводится в тексте только один раз.\\
\indent\textit{Введения аббревиатур} имеют определенную структуру, которая задается формулой (таблица 1) [3]. Возможны ситуации, когда при введении аббревиатуры в скобках может присутствовать текст, который не относится ни к расшифровке, ни к аббревиатуре.\\
\begin{center}
    \begin{tabular}{|p{5cm}|p{6cm}|}
        \hline
        Формула введения & Примеры \\
        \hline
        расшифровка (аббревиатура) & Специальное программное обеспечение (СПО), система (С)\\
        \hline
        аббревиатура (расшифровка) & АС (автоматическая сигнализация), СО (сигнал ожидания)\\
        \hline
        расшифровка аббревиатура & Процессор преобразования матриц ППМ, двухпроцессорная система ДС\\
        \hline
        аббревиатура – расшифровка & ЯМД – язык манипулирования данными, ЯУ – язык управления\\
        \hline
        (расшифровка – аббревиатура)1 & (Главная машина – ГМ) \\
        \hline
    \end{tabular}
\end{center}
\\
\indent\textit{Поиск аббревиатур без расшифровок} заключается в определении по определенным признакам наличия аббревиатур в предложении.\\
\indentВ процессе поиска могут встречаться аббревиатуры, которые ранее не были введены в тексте. В связи с этим появляется необходимость поиска информации об их расшифровках по другим источникам. Для корректного сопоставления аббревиатуры и расшифровки из разных текстов, необходимо учитывать контекст аббревиатур (рисунок 3).
\begin{figure}[!h]
    \begin{center}
        \includegraphics[scale=0.5]{image2.jpg}
        \caption{Соответствие расшифровки, аббревиатуры и контекста\\ \centering{(разработано авторами)}}
        \label{claster_pic1}
    \end{center}
\end{figure}
\\
\indentТак как одна аббревиатура вводится в тексте, как правило, только один раз, то далее по тексту будут встречаться только аббревиатуры без расшифровок, при этом каждая выделенная аббревиатура будет однозначно соответствовать введенной ранее. Далее от предложения к предложению необходимо считывать аббревиатуры и их контекст в базу данных, после чего производить сопоставление с информацией, полученной при \textit{поиске мест введения аббревиатур}. В случае отсутствия расшифровок необходимо производить поиск по другим текстам или по словарю.\\
\indentПринципиальная схема ВАиР в тексте приведена на рисунке 4. \\
\begin{center}
    \includegraphics[scale=0.5]{image3.jpg}\\
    \caption{Рис.4: Принципиальная схема выделения аббревиатур и расшифровок в тексте (разработано авторами)}
\end{center}
\\
\indent\textbf{5. Заключение}\\
\indentВ статье рассматривается обобщенное гиперболическое уравнение запаздывающего типа с некарлемановскими сдвигами вида\\
$$
U_x_x(x,y) - U_y_y(x,y) = H(x- \tau) [U_x(x- \tau .y)+ U(x- \tau .y)], \eqno {(1)}
$$
в области\\
$$
D_k = \{(x,y) : x > 0, y < o\} = \displaystyle\bigcup_{k=o}^{\infty}D_k,
$$
где\\
$$
D_k = \{(x,y):k\tau-y \leq x \leq (k+1) \tau +y , -(\tau/2) < y < o\} (k=0,1,2,...)
$$
\textbf{Задача К} Найди в области $D$  решение уравнения из класса  $ C(\overline D) \cap  C^2 (D)$, удовлетворяющее условиям \\
$
U(x,y)|_y_=_o=\omega(x),x \geq 0, 
$ \\
$
U_y(x,y)|_y_=_o=u(x),x > 0, 
$ \\
где $\omega (x), u(x) -$ заданные непрерывные достаточно гладкие функции    , причем $ \omega (o) = \omega (+\infty) =0$
\begin{center}
    \textbf{Список литературы}
\end{center} \\
1. Максименко, О.И. Новые тенденции аббревиации (на материале русского, английского и немецкого языков) // Вестник Российского университета дружбы народов. Серия: Теория языка. Семиотика. Семантика. – 2017. – Т. 8. – № 1. – С. 174-181.\\
2. Грязнухина, Т.А. Лингвистические проблемы автоматизации редакционно-издательских процессов / Т.А. Грязнухина, Н.П. Дарчук, Л.И. Комарова и др.; отв. ред. В.И. Перебейнос, М.Д. Феллер // колл. монография: Академия наук УССР, Институт языковедения им. А.А. Потебни. — Киев: Наукова думка, 1986. – 229 с.\\
3. Гращенко, Л.А. Информационные основы польско-русского межъязыкового преобразования текстов / Л.А. Гращенко, Н.Н. Пивоваров // Новые информационные технологии в автоматизированных системах. – 2016. – № 19. – С. 101-106.\\
4. Мизернов, И.Ю. Анализ методов оценки сложности текста / И.Ю. Мизернов, Л.А. Гращенко // Новые информационные технологии в автоматизированных системах. – 2015. – № 18. – С. 572-581.\\
5. Ванюшкин, А.С. Методы и алгоритмы извлечения ключевых слов / А.С. Ванюшкин, Л.А. Гращенко // Новые информационные технологии в автоматизированных системах. – 2016. – № 19. – С. 85-93.
6. Науменко, Д.А. Информационные основы автоматизации рерайтинга / Д.А. Науменко, Л.А. Гращенко, Г.В. Романишин // Новые информационные технологии в автоматизированных системах. – 2019. – № 22. – С. 187-191.\\
7. Гращенко, Л.А. Опыт автоматизированного анализа повторов в научных текстах / Л.А. Гращенко, Г.В. Романишин // Новые информационные технологии в автоматизированных системах. – 2015. – № 18. – С. 582-590.\\
8. Суперанская, А.В. Общая терминология: Вопросы теории. Аббревиация в терминологии / А.В. Суперанская, Н.В. Подольская, Н.В. Васильева. – Изд. 6-е. – М.: Книжный дом «ЛИБРОКОМ», 2012. – 248 с.\\
9. Нургалеева, Т.Г. Аббревиация как средство экспрессивного словообразования: автореф. дис. канд. филол. наук, спец. 10.02.04 «Германские языки» / Т. Г. Нургалеева. — М.: Наука, 2010. — 240 с.\\
10. Земская, Е.А. Современный русский язык. Словообразование: учеб. Пособие. 3-е изд., испр. и доб. – М.: Наука, 2011.\\
11. Jenkins K. Deciphering Journal Abbreviations with JAbbr // Code4Lib Journal. – 2009. – № 7. [Электронный ресурс] URL:https://journal.code4lib.org/articles/1758 (Дата обращения: 09.11.2021)\\
12. Mikolov T. Efficient Estimation of Word Representations in Vector Space / T. Mikolov, K. Chen, G. Corrado, J. Dean // arXiv.org. — 2013. [Электронный ресурс] URL:http://arxiv.org/pdf/1301.3781v3.pdf (Дата обращения: 10.11.2021)\\
13. Mikolov T. Distributed Representations of Words and Phrases and their Compositionality / T. Mikolov, I. Sutskever, K. Chen, G. Corrado, J. Dean // Advances in Neural Information Processing Systems. – 2013. – P. 3111-3119.\\
14. Blei, D.M. Latent Dirichlet Allocation / D.M. Blei, A.Y. Ng, M.I. Jordan // Journal of Machine Learning Research. – 2003. – № 3. – P. 993-1022.\\
15. Heinrich G. Parameter estimation for text analysis. — 2004. [Электронный ресурс] URL:http://citeseerx.ist.psu.edu/view-doc/summary?doi=10.1.1.216.695 (Дата обращения: 10.11.2021)\\
16. Автоматизированное формирование перечня аббревиатур (сокращений) / А.А. Чумичкин, М.В. Рутц, Г.И. Трифонов // Свидетельство о регистрации программы для ЭВМ RU 2018663069, 19.10.2018. Заявка № 2018660897 от 04.10.2018.\\
17. Программа для извлечения и анализа терминологических структур смежных предметных областей / Д.А. Губанов // Свидетельство о регистрации программы для ЭВМ RU 2019665358, 22.11.2019. Заявка № 2018664640 от 18.11.2019.\\
18. Алексеев, Д.И. Сокращённые слова в русском языке. – Саратов: Изд-во Саратовского ун-та, 1979. – 328 с.\\
19. Виноградова, В.В. Русская грамматика: научные труды. В 2 т. Т. 1 – М.: Российская академия наук, 2005. — 784 с.\\
20. Ахманова, О.С. Словарь лингвистических терминов. – М.: Едиториа УРСС, 2004. – 576 с.
\end{document}
